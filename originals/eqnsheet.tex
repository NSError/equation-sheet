							% %%%%%%%%%%%%%%%%%%%%%%%%%%%%%%%%%%%%%%%%%%%%%%%%% %
							% Document Created 10 October 2009 by Chris Miller. %
							% Copyright 2009 FSDEV.  All rights reserved.       %
							% Academic endorsement.                             %
							% %%%%%%%%%%%%%%%%%%%%%%%%%%%%%%%%%%%%%%%%%%%%%%%%% %
 
\documentclass[10pt,letterpaper]{article}
\usepackage[utf8x]{inputenc}
\usepackage{ucs}
\usepackage{amsmath}
\usepackage{amsfonts}
\usepackage{amssymb}
\usepackage{pst-all}
\usepackage{multicol}
\usepackage[left=0.5in,top=0.5in,right=0.5in,nohead,nofoot]{geometry}
\pagestyle{empty}
\begin{document}
\title{Equation Sheet}
\author{Chris Miller}

\newcommand{\sech}{\text{sech}}
\newcommand{\csch}{\text{csch}}

															% %%%%%%%%%%%%%%%%%%%%%%%%%%%%%%%%%%%%%%%%%%%%%%%%% %
															\subsection*{Trigonometric Formulas}
															% %%%%%%%%%%%%%%%%%%%%%%%%%%%%%%%%%%%%%%%%%%%%%%%%% %
\begin{center}\begin{minipage}[center]{340pt}
\begin{minipage}[left]{200pt}
\begin{pspicture}(-1.25,-1.25)(1.25,1.25)
\pscircle(0,0){1}
\psaxes[labels=none]{<->}(0,0)(-1.25,-1.25)(1.25,1.25)
\psline{-}(0,0)(0.72,0.72)
\psarc{->}(0,0){0.77}{0}{45}
\rput(0.45,0.2){$\theta$}
\rput(0.25,0.5){$r$}
\psdot(0.72,0.72)
\rput(1.2,0.9){$\left( x, y \right)$}
\end{pspicture}
\end{minipage}
\begin{minipage}[right]{120pt}
$ \begin{array}{l}
				  \sin \theta = \frac{y}{r} \\
				  \cos \theta = \frac{x}{r} \\
				  \tan \theta = \frac{y}{x}
				 \end{array} $ \hfill
\end{minipage}\end{minipage}\end{center}

\begin{multicols}{2}											\subsubsection*{Reciporicals}

$ \cot \theta = \frac{1}{\tan \theta} $ \hfill
$ \sec \theta = \frac{1}{\cos \theta} $ \hfill
$ \csc \theta = \frac{1}{\sin \theta} $

															\subsubsection*{Definitions}

$ \cot \theta = \frac{\cos \theta}{\sin \theta} $ \hfill
$ \sec \theta = \frac{1}{\cos \theta} $ \hfill
$ \csc \theta = \frac{1}{\sin \theta} $

															\subsubsection*{Hyperbolic}
															
$ \sinh \theta = \frac{ e ^{\theta} - e ^{-\theta} }{ 2 } $ \hfill
$ \cosh \theta = \frac{ e ^{\theta} + e ^{-\theta} }{ 2 } $ \hfill
$ \tanh \theta = \frac{ e ^{\theta} - e ^{-\theta} }{ e ^{\theta} + e ^{-\theta} } $

															\subsubsection*{Pythagorean}

$ \sin ^{2} \theta + \cos ^{2} \theta = 1 $ \hfill
$ \tan ^{2} \theta + 1 = \sec ^{2} \theta $ \hfill
$ 1 + \cot ^{2} \theta = \csc ^{2} \theta $

															\subsubsection*{Cofunction}

$ \sin (\frac{\pi}{2} - \theta ) = \cos \theta $ \hfill
$ \cos (\frac{\pi}{2} - \theta ) = \sin \theta $ \hfill
$ \tan (\frac{\pi}{2} - \theta ) = \cot \theta $

															\subsubsection*{Even/Odd}

$ \sin -\theta = -\sin \theta $ \hfill
$ \cos -\theta = \cos \theta $ \hfill
$ \tan -\theta = -\tan \theta $

															\subsubsection*{Double Angle}

\begin{center} $ \begin{array}{l}
				  \sin 2\theta = 2\sin\theta\cos\theta \\
				  \cos 2\theta = \cos ^{2} \theta - \sin ^{2} \theta \\
				  \cos 2\theta = 1 - 2\sin ^{2}\theta
				 \end{array} $ \end{center}

															\subsubsection*{Half-Angle}

$ \sin ^{2} \theta = \frac{1 - \cos 2 \theta}{2} $ \hfill
$ \cos ^{2} \theta = \frac{1+\cos 2\theta}{2} $

															\subsubsection*{Addition}

\[ \sin ( a + b ) = \sin a\cos b + \cos a\sin b \]
\[ \cos ( a + b ) = \cos a\cos b - \sin a\sin b \]

															\subsubsection*{Subtraction}

\[ \sin ( a - b ) = \sin a\cos b - \cos a\sin b \]
\[ \cos ( a - b ) = \cos a\cos b + \sin a\sin b \]

															\subsubsection*{Sum}

\[ \sin u + \sin v = 2\sin \frac{u+v}{2}\cos \frac{u-v}{2} \]
\[ \cos u + \cos v = 2\cos \frac{u+v}{2}\cos \frac{u-v}{2} \]

															\subsubsection*{Product}

\[ \sin u\sin v = \frac{1}{2}\left[ \cos(u-v)-\cos(u+v) \right] \]
\[ \cos u\cos v = \frac{1}{2}\left[ \cos(u-v)+\cos(u+v) \right] \]
\[ \sin u\cos v = \frac{1}{2}\left[ \sin(u+v)+\sin(u-v) \right] \]
\[ \cos u\sin v = \frac{1}{2}\left[ \sin(u+v)-\sin(u-v) \right] \]

\end{multicols}												\subsubsection*{Unit Circle}

\begin{center}\begin{minipage}[center]{400pt}\begin{minipage}[left]{200pt}
\begin{pspicture}(-2,-2)(2,2)
\pscircle(0,0){1.75}
\psaxes[labels=none]{<->}(0,0)(-2,-2)(2,2)
\psdot(1.75,0) \rput[lb](1.85,0.1){\small$0$}
\psdot(1.5652476,0.78262379) \rput[lb](1.6652476,0.88262379){$\frac{\pi}{6}$}
\psdot(1.2374369,1.2374369) \rput[lb](1.3374369,1.3374369){$\frac{\pi}{4}$}
\psdot(0.78262379,1.5652476) \rput[lb](0.88262379,1.6652476){$\frac{\pi}{3}$}
\psdot(0,1.75) \rput[lb](0.1,1.85){$\frac{\pi}{2}$}
\psdot(-0.78262379,1.5652476) \rput[rb](-0.88262379,1.6652476){$\frac{2\pi}{3}$}
\psdot(-1.2374369,1.2374369) \rput[rb](-1.3374369,1.3374369){$\frac{3\pi}{4}$}
\psdot(-1.5652476,0.78262379) \rput[rb](-1.6652476,0.88262379){$\frac{5\pi}{6}$}
\psdot(-1.75,0) \rput[rb](-1.85,0.1){\small$\pi$}
\end{pspicture}
\end{minipage}
\begin{minipage}[right]{180pt}
\begin{tabular}{llll}
$\sin 0 =$ & $0$							&$\cos 0 =$ & $1$										\\
$\sin\frac{\pi}{6} =$ & $\frac{1}{2}$		&$\cos \frac{\pi}{6} =$ & $\frac{\sqrt{3}}{2}$			\\
$\sin\frac{\pi}{4} =$ & $\frac{\sqrt{2}}{2}$
											&$\cos \frac{\pi}{4} =$ & $\frac{\sqrt{2}}{2}$			\\
$\sin\frac{\pi}{3} =$ & $\frac{\sqrt{3}}{2}$
											&$\cos \frac{\pi}{3} =$ & $\frac{1}{2}$					\\
$\sin\frac{\pi}{2} =$ & $1$					&$\cos \frac{\pi}{2} =$ & $0$							\\
$\sin\frac{2\pi}{3} =$ & $\frac{\sqrt{3}}{2}$
											&$\cos \frac{2\pi}{3} =$ & $-\frac{1}{2}$				\\
$\sin\frac{3\pi}{4} =$ & $\frac{\sqrt{2}}{2}$
											&$\cos \frac{3\pi}{4} =$ & $-\frac{\sqrt{2}}{2}$		\\
$\sin\frac{5\pi}{6} =$ & $\frac{1}{2}$		&$\cos \frac{5\pi}{6} =$ & $-\frac{\sqrt{3}}{2}$		\\
$\sin\pi =$ & $0$							&$\cos\pi =$ & $-1$										\\
\end{tabular}
\end{minipage}\end{minipage}\end{center}
															% %%%%%%%%%%%%%%%%%%%%%%%%%%%%%%%%%%%%%%%%%%%%%%%%% %
\pagebreak\begin{multicols}{2}								\subsection*{Derivative Formulas}
															% %%%%%%%%%%%%%%%%%%%%%%%%%%%%%%%%%%%%%%%%%%%%%%%%% %
															\subsubsection*{General Rules}

\begin{center} $ \begin{array}{l}
   \frac{d}{dx} \left[ f(x) \pm g(x) \right] = f ^\prime (x) \pm g ^\prime (x) \\
   \frac{d}{dx} \left[ f \left( g(x) \right) \right] = f ^\prime  \left( g(x) \right) g ^\prime (x) \\
   \frac{d}{dx} \left[ cf(x) \right] = cf ^\prime (x) \\
   \frac{d}{dx} \left[ f(x)g(x) \right] = f ^\prime (x)g(x) + f(x)g ^\prime (x) \\
   \frac{d}{dx} \left[ \frac{f(x)}{g(x)} \right] = \frac{f ^\prime (x)g(x)-f(x)g ^\prime (x)}{ \left[ g(x) \right] ^{2}}
  \end{array} $ \end{center}		
															\subsubsection*{Power Rules}

$ \frac{d}{dx}\: x^{r} = rx^{r - 1} $ \hfill
$ \frac{d}{dx}\: c = 0 $ \hfill
$ \frac{d}{dx}\: cx = c $ \hfill
$ \frac{d}{dx}\: \sqrt{x} = \frac{1}{2 \sqrt{x} } $

															\subsubsection*{Exponential Rules}

$ \begin{array}{ll}
   \frac{d}{dx}\: e^{x} = e^{x} \\
   \frac{d}{dx}\: e^{u(x)} = e^{u(x)}u ^\prime (x)
  \end{array} $ \hfill
$ \begin{array}{ll}
   \frac{d}{dx}\: a^{x} = a^{x} \ln a \\
   \frac{d}{dx}\: e^{rx} = re^{rx}
  \end{array} $ \hfill
$ \frac{d}{dx} \ln x = \frac{1}{x} $

															\subsubsection*{Trigonometric Rules}

$ \begin{array}{ll}
   \frac{d}{dx}\: \sin x = \cos x \\
   \frac{d}{dx}\: \tan x = \sec ^{2} x \\
   \frac{d}{dx}\: \sec x = \sec x \tan x
  \end{array} $ \hfill
$ \begin{array}{ll}
   \frac{d}{dx}\: \cos x = -\sin x \\
   \frac{d}{dx}\: \cot x = -\csc ^{2} x \\
   \frac{d}{dx}\: \csc x = -\csc x \cot x
  \end{array} $

															\subsubsection*{Inverse Trigonometric Rules}

$ \begin{array}{ll}
   \frac{d}{dx}\: \sin ^{-1} x = \frac{1}{\sqrt{1 - x ^{2} }} \\
   \frac{d}{dx}\: \tan ^{-1} x = \frac{1}{1 + x ^ {2}} \\
   \frac{d}{dx}\: \sec ^{-1} x = \frac{1}{|x|\sqrt{x^{2}-1}}
  \end{array} $ \hfill
$ \begin{array}{ll}
   \frac{d}{dx}\: \cos ^{-1} x = \frac{-1}{\sqrt{1 - x ^{2} }} \\
   \frac{d}{dx}\: \cot ^{-1} x = \frac{-1}{1 + x ^ {2}} \\
   \frac{d}{dx}\: \csc ^{-1} x = \frac{-1}{|x|\sqrt{x^{2}-1}}
  \end{array} $

															\subsubsection*{Hyperbolic Rules}

$ \begin{array}{ll}
   \frac{d}{dx}\: \sinh x = \cosh x \\
   \frac{d}{dx}\: \tanh x = \sech x \\
   \frac{d}{dx}\: \sech x = -\sech x \tanh x
  \end{array} $ \hfill
$ \begin{array}{ll}
   \frac{d}{dx}\: \cosh x = \sinh x \\
   \frac{d}{dx}\: \coth x = -\csch ^{2} x \\
   \frac{d}{dx}\: \csch x = -\csch x \coth x
  \end{array} $

															\subsubsection*{Inverse Hyperbolic Rules}
															
$ \begin{array}{ll}
   \frac{d}{dx}\: \sinh ^{-1} x = \frac{1}{\sqrt{1 + x ^{2}}} \\
   \frac{d}{dx}\: \tanh ^{-1} x = \frac{1}{1 - x ^{2}} \\
   \frac{d}{dx}\: \sech ^{-1} x = \frac{-1}{x \sqrt{1 - x ^{2}}}
  \end{array} $ \hfill
$ \begin{array}{ll}
   \frac{d}{dx}\: \cosh ^{-1} x = \frac{1}{\sqrt{x ^{2} - 1}} \\
   \frac{d}{dx}\: \coth ^{-1} x = \frac{1}{1 - x ^{2}} \\
   \frac{d}{dx}\: \csch ^{-1} x = \frac{-1}{|x|\sqrt{x ^{2} + 1}}
  \end{array} $

\end{multicols}
															% %%%%%%%%%%%%%%%%%%%%%%%%%%%%%%%%%%%%%%%%%%%%%%%%% %
															\subsection*{Methods}
															% %%%%%%%%%%%%%%%%%%%%%%%%%%%%%%%%%%%%%%%%%%%%%%%%% %
\begin{multicols}{2}

															\subsubsection*{Linear Approximation}

The linear approximation of $f(x)$ at $x=x_{0}$ is
\[ L(x) = f(x_{0}) + f^\prime (x_{0})(x-x_{0}) \]

															\subsubsection*{Newton's Method}

$f(x_{n})\approx 0$ for

\[ x_{n+1} = x_{n} - \frac{f(x_{n})}{f^\prime (x_{n})}, n=0,1,2,3,... \]

\end{multicols}
\end{document}